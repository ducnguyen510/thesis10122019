\documentclass[13pt,a4paper,vietnamese]{report}
\usepackage{babel}%sử dụng tiếng việt
\usepackage[margin=1in]{geometry}
\usepackage{amsfonts,amsmath,amssymb}
\usepackage[none]{hyphenat}
\usepackage{fancyhdr}
\usepackage{graphicx}
\usepackage{float}
\usepackage{subcaption}
\usepackage[nottoc,notlot,notlof]{tocbibind}
\usepackage{titlesec}
\usepackage{makeidx}

\pagestyle{fancy}
\fancyhead{}
\fancyfoot{}
\fancyhead[L]{\slshape\MakeUppercase{Floating point}}
\fancyhead[R]{\slshape{Student's Name}}
\fancyfoot[C]{\thepage}
%\renewcommand{\headrulewidth}{0pt}
\renewcommand{\footrulewidth}{0pt}

%\parindent 0ex
%\setlength{\parindent}{4ex}
%\setlength{\parskip}{1ex}
\renewcommand{\baselinestretch}{1.5}
\makeindex
\begin{document}
\begin{titlepage}
\begin{center}
\vspace*{1cm}
\Large\textbf{Floating point}\\
\large\textbf{Viettel IC Design}\\
\vfill
\line(1,0){400}\\[1mm]
\huge{\textbf{Floating point}}\\[3mm]
\Large{\textbf{-Foating point based on FPGA-}}\\[1mm]
\line(1,0){400}\\
\vfill
By Van Duc NGUYEN\\
Candidate 20151050 \\
\today
\end{center}
\end{titlepage}
	
\tableofcontents

\thispagestyle{empty}
\clearpage

\setcounter{page}{1}
\chapter{Introduction}
Trong biểu diễn số học trong máy tính có 2 kiểu biểu diễn phổ biến,đó là kiểu fixed-point và kiểu floating-point.Với kiểu biểu diễn fixed-point(ví dụ:mã bù 2),hoàn toàn có thể biểu diễn các số nguyên dương và ân hoặc các số gần 0.Bằng cách cố định radix 2 hoặc radix cao hơn.Với kiểu định dạng này,có thể biểu diễn số với phần thập phân 1 cách dễ dàng.

Tuy nhiên,cách tiếp cận này có vài hạn chế.Không thể biểu diễn các số rất lớn hoặc các số rất nhỏ.Hơn nữa,phần thập phân của thương của 1 phép chia 2 số có thể bị mất mát.
Ví dụ với hệ thập phân,với số 976,000,000,000,000 có thể biểu diễn thành $9.76*10^{14}$,bằng cách biểu diễn như vậy,
\chapter{FLoating-point representation}

\chapter{IEEE 754/2008 standard}

\chapter{Floating-point arthmetic}

\section{Addision and Subtraction}
\section{Multiplication and Divison}
\section{Reciprocal and Square Root}

\chapter{Implement in FPGA}

\chapter{Evaluation}

\chapter{Conclussion and Related-work}
This is end of document

\cite{name1}\\

\pagebreak
\begin{thebibliography}{}
\bibitem{name1}
Duc Nguyen,"sach cua tao",\textit{bai bao hoi nghi} Sun,25,May,2019.
\bibitem{name2}
Authors
\textit{"title of article"}
Title of Journal
Date,Month,Year
page
Medium of publish
\end{thebibliography}
\end{document}